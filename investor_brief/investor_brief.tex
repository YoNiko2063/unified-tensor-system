\documentclass[10pt,letterpaper]{article}

\usepackage{geometry}
\geometry{
  top=0.75in, bottom=0.75in,
  left=0.85in, right=0.85in
}

\usepackage{amsmath,amssymb}
\usepackage{graphicx}
\usepackage{booktabs}
\usepackage{xcolor}
\usepackage{microtype}
\usepackage{enumitem}
\usepackage{multicol}
\usepackage{hyperref}
\usepackage{titlesec}
\usepackage{mdframed}
\usepackage{array}
\usepackage{tabularx}

% Colors
\definecolor{accent}{RGB}{20, 60, 120}
\definecolor{lightgray}{RGB}{245, 245, 245}
\definecolor{midgray}{RGB}{100, 100, 100}

% Section formatting
\titleformat{\section}{\large\bfseries\color{accent}}{}{0em}{}[\color{accent}\titlerule]
\titleformat{\subsection}{\normalsize\bfseries\color{accent}}{}{0em}{}
\titlespacing{\section}{0pt}{10pt}{4pt}
\titlespacing{\subsection}{0pt}{6pt}{2pt}

\setlength{\parindent}{0pt}
\setlength{\parskip}{3pt}
\setlist[itemize]{leftmargin=1.2em, itemsep=1pt, topsep=2pt, parsep=0pt}

\hypersetup{colorlinks=true, linkcolor=accent, urlcolor=accent, hidelinks}
\pagestyle{empty}

% ─────────────────────────────────────────────────────────────────────────────
\begin{document}

% ── HEADER ───────────────────────────────────────────────────────────────────
\begin{center}
  {\LARGE\bfseries\color{accent} Unified Tensor Systems}\\[4pt]
  {\large Analytic Stability Acceleration for Nonlinear Dynamical Networks}\\[3pt]
  {\small\color{midgray}\textit{Patent Pending} \quad|\quad
   \texttt{yoonikolas@gmail.com}}
\end{center}

\vspace{2pt}
\noindent\color{accent}\rule{\linewidth}{1.2pt}
\vspace{4pt}

% ── HEADLINE BOX ─────────────────────────────────────────────────────────────
\begin{mdframed}[
  backgroundcolor=lightgray,
  linecolor=accent,
  linewidth=1pt,
  innerleftmargin=10pt, innerrightmargin=10pt,
  innertopmargin=7pt, innerbottommargin=7pt
]
\begin{center}
{\Large\bfseries 57{,}946$\times$ faster} \quad
{\large than RK4 time-domain integration}\\[3pt]
{\normalsize IEEE 39-bus New England System \quad|\quad
 ${<}2.73\%$ deviation from reference \quad|\quad
 Patent Pending}
\end{center}
\end{mdframed}

\vspace{6pt}

% ── TWO-COLUMN BODY ──────────────────────────────────────────────────────────
\begin{multicols}{2}

\section*{The Problem}

Modern power systems rely on time-domain integration for stability assessment.
Critical clearing time (CCT) estimation --- determining the maximum fault
duration before a generator loses synchronism --- is the central bottleneck
in N-1 contingency screening.

A 500-bus system with 50 generators and 500 contingencies requires
\textbf{25{,}000 CCT computations.}
The standard RK4 binary-search method demands approximately
$13 \times 3{,}000 \times 4 = 156{,}000$
ODE evaluations per CCT.
At 0.1\,ms per evaluation, a full N-1 screen takes \textbf{over four hours.}

Real-time contingency analysis, energy market operation, and remedial action
schemes all require orders-of-magnitude improvement.

\section*{The Breakthrough}

A proprietary analytic method that replaces brute-force time integration
with direct computation from the geometry of the stability manifold.

Rather than simulating trajectories step-by-step, the method identifies
stability boundaries \textbf{analytically} --- requiring \textbf{zero ODE
evaluations} per CCT estimate.

The approach is validated on the IEEE 39-bus standard benchmark, patent
protected, and applicable to any nonlinear dynamical network.

\section*{Validation}

\begin{tabularx}{\columnwidth}{lX}
\toprule
\textbf{Benchmark} & IEEE 39-bus New England System \\
\midrule
Generators & 10 \\
Speedup    & \textbf{57,946$\times$} over RK4 \\
Max error  & \textbf{$<$2.73\%} deviation \\
Damping    & $\zeta = 0.00$--$0.20$ (full realistic range) \\
Cross-val  & $\binom{10}{2} = 45$ generator subsets \\
Stability  & Universal parameter, 2.3\% range across all splits \\
Test suite & 2,239 automated tests passing \\
\bottomrule
\end{tabularx}

\vspace{4pt}

The key parameter is universal across all generator subsets ---
a \textbf{structural property} of the system, not a curve-fitting artifact.
A network with 1{,}000 generators uses the same value, pre-computed once.

\section*{Platform Architecture}

The engine operates as a drop-in acceleration layer for existing stability
workflows:

\begin{itemize}
  \item Accepts standard network parameters as input
  \item Returns CCT estimates with bounded error guarantees
  \item Integrates with existing simulation platforms via API or plugin
  \item Scales linearly with network size
  \item No ODE solver required at inference time
\end{itemize}

A multi-objective interface enables simultaneous targeting of stability
margins, component cost, and robustness --- with Pareto front and
basin-of-attraction analysis available.

\section*{Applications}

\textbf{Power Systems}
\begin{itemize}
  \item Real-time N-1 contingency screening
  \item Renewable integration stability envelopes
  \item Online stability monitoring for grid operators
  \item Inverter and grid-forming control validation
\end{itemize}

\textbf{Power Electronics}
\begin{itemize}
  \item Converter stability margins
  \item Switching transient analysis
\end{itemize}

\textbf{Multiphysics Simulation}
\begin{itemize}
  \item Accelerated PDE stability screening
  \item Structural and thermal transition analysis
\end{itemize}

\section*{Market Opportunity}

N-1 contingency screening is a mandatory regulatory requirement for
transmission operators in all major electricity markets.
The power systems simulation software market serves thousands of utilities,
grid operators, independent system operators, and engineering consultancies
globally.

Accelerating stability computation by orders of magnitude opens use cases
previously out of reach: real-time operation, tighter renewable integration
margins, and automated remedial action schemes --- all requiring sub-second
stability decisions.

\section*{Commercial Model}

The acceleration engine is offered as a licensed module for integration into
existing simulation platforms, or as a standalone API for direct deployment
by grid operators and engineering firms.

\begin{itemize}
  \item \textbf{Platform licensing} --- embedded in PSS/E, PowerFactory,
        PSCAD, or equivalent workflows
  \item \textbf{API licensing} --- direct access for ISO/RTO and utility
        operations teams
  \item \textbf{Pilot deployment} --- scoped N-1 screening integration with
        measurable throughput improvement
\end{itemize}

\section*{Competitive Advantage}

\begin{tabularx}{\columnwidth}{lX}
\toprule
 & \textbf{Approach} \\
\midrule
Traditional & Time-domain integration, brute-force iteration \\
Ours        & Proprietary analytic method, stability manifold geometry \\
\midrule
 & \textbf{Advantage} \\
\midrule
Speed     & Orders-of-magnitude acceleration \\
Accuracy  & Bounded error, validated under realistic damping \\
Scale     & Linear scaling, real-time capable \\
IP        & Provisional USPTO patent filed \\
\bottomrule
\end{tabularx}

\section*{Intellectual Property}

A provisional patent has been filed with the USPTO covering the core
analytic acceleration method and its application to nonlinear dynamical
networks.

Method details are confidential and available under NDA.

\section*{Vision}

To become the \textbf{foundational acceleration layer} for stability-critical
simulation in power systems and nonlinear dynamical infrastructure ---
enabling real-time contingency analysis at scales previously impossible.

\vspace{6pt}
\noindent\color{accent}\rule{\columnwidth}{0.6pt}\\[3pt]
{\small\color{midgray}
For collaboration, licensing, or pilot deployments:\\
\texttt{yoonikolas@gmail.com} \quad|\quad Patent Pending
}

\end{multicols}

\end{document}
